\documentclass[12pt]{article}
 
\usepackage[margin=1in]{geometry} 
\usepackage{amsmath,amsthm,amssymb}
\usepackage{gensymb}
\usepackage{hyperref}
\usepackage{graphicx}
\usepackage{indentfirst}
 

\begin{document}

\date{\today}
\title{Proper Motions, Color-Magnitude Diagrams and Luminosity Functions with HST Images}
\author{Sean Terry \& Ishaan Gandhi} 
 
\maketitle

\section*{\centering{Procedure}}
Two epochs (2010 and 2012) of HST images were retrieved from MAST (proposal ID: 11664, 12666), the images are exactly two years apart and taken at the same telescope orientation. The filters and channels used are; F555W, F814W, F110W, F160W for the 2010 data and F814W for the 2012 data, all from WFC3 UVIS/IR.
\\
\\
\indent Using the photometry routine DOLPHOT, a master catalog of stellar magnitudes and positions was produced. This catalog matches stars across the 2010 dataset filters (F555, 814, 110, 160) using the 814 drizzled reference image to map the pixel positions. A separate run of Dolphot was initiated for the 2012 dataset F814W images using the corresponding drizzled reference image for that epoch. There is a functionality in DOLPHOT that allows for high accuracy astrometry using the Anderson PSF cores for UVIS/IR; this was enabled during the two separate runs that produced the 2010 master catalog and the 2012 F814W catalog. The 2010 master catalog output from Dolphot has 84,486 stars, which is matched to the 2012 catalog (173,273 stars) to produce a matched catalog of 65,030 stars. During the Dolphot run and subsequent matching, several rejections were made ('non-stellar' object flag in Dolphot, oversaturated stars that made it to the output catalog, magnitude uncertainty cuts `sigma-clipping').
\\
\\
\indent Due to uncertainty in the absolute pointing of HST, there is a systematic shift of all objects between the drizzled reference frames in the two epochs. This shift is approximately three pixels in the x direction and one pixel in the y direction on the detector. The average of these shifts was calculated and then applied to each object mapped to the 2010 drizzled frame, this leaves the relative proper motions of the stars (\textbf{with some associated uncertainty here???}). MORE STUFF ABOUT CHOOSING THE WINDOW CUT BETWEEN DISK/BULGE BRANCH IN THE CMD TO GET PROPER MOTIONS OF EACH POPULATION AND EXTRAPOLATING DOWN BELOW THE MSTO, ETC.
\\
\\
\indent A clean luminosity function was then created from the PM cleaned catalog. The shape of the LF agrees with literature (Clarkson 08, Calamida 14). This process was repeated for the deeper IR object lists (F110W - F160W)



 
\end{document}